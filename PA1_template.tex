% Options for packages loaded elsewhere
\PassOptionsToPackage{unicode}{hyperref}
\PassOptionsToPackage{hyphens}{url}
%
\documentclass[
]{article}
\usepackage{amsmath,amssymb}
\usepackage{lmodern}
\usepackage{ifxetex,ifluatex}
\ifnum 0\ifxetex 1\fi\ifluatex 1\fi=0 % if pdftex
  \usepackage[T1]{fontenc}
  \usepackage[utf8]{inputenc}
  \usepackage{textcomp} % provide euro and other symbols
\else % if luatex or xetex
  \usepackage{unicode-math}
  \defaultfontfeatures{Scale=MatchLowercase}
  \defaultfontfeatures[\rmfamily]{Ligatures=TeX,Scale=1}
\fi
% Use upquote if available, for straight quotes in verbatim environments
\IfFileExists{upquote.sty}{\usepackage{upquote}}{}
\IfFileExists{microtype.sty}{% use microtype if available
  \usepackage[]{microtype}
  \UseMicrotypeSet[protrusion]{basicmath} % disable protrusion for tt fonts
}{}
\makeatletter
\@ifundefined{KOMAClassName}{% if non-KOMA class
  \IfFileExists{parskip.sty}{%
    \usepackage{parskip}
  }{% else
    \setlength{\parindent}{0pt}
    \setlength{\parskip}{6pt plus 2pt minus 1pt}}
}{% if KOMA class
  \KOMAoptions{parskip=half}}
\makeatother
\usepackage{xcolor}
\IfFileExists{xurl.sty}{\usepackage{xurl}}{} % add URL line breaks if available
\IfFileExists{bookmark.sty}{\usepackage{bookmark}}{\usepackage{hyperref}}
\hypersetup{
  pdftitle={PA1\_template.Rmd},
  hidelinks,
  pdfcreator={LaTeX via pandoc}}
\urlstyle{same} % disable monospaced font for URLs
\usepackage[margin=1in]{geometry}
\usepackage{graphicx}
\makeatletter
\def\maxwidth{\ifdim\Gin@nat@width>\linewidth\linewidth\else\Gin@nat@width\fi}
\def\maxheight{\ifdim\Gin@nat@height>\textheight\textheight\else\Gin@nat@height\fi}
\makeatother
% Scale images if necessary, so that they will not overflow the page
% margins by default, and it is still possible to overwrite the defaults
% using explicit options in \includegraphics[width, height, ...]{}
\setkeys{Gin}{width=\maxwidth,height=\maxheight,keepaspectratio}
% Set default figure placement to htbp
\makeatletter
\def\fps@figure{htbp}
\makeatother
\setlength{\emergencystretch}{3em} % prevent overfull lines
\providecommand{\tightlist}{%
  \setlength{\itemsep}{0pt}\setlength{\parskip}{0pt}}
\setcounter{secnumdepth}{-\maxdimen} % remove section numbering
\ifluatex
  \usepackage{selnolig}  % disable illegal ligatures
\fi

\title{PA1\_template.Rmd}
\author{}
\date{\vspace{-2.5em}}

\begin{document}
\maketitle

\texttt{\{r\ setup,\ include=FALSE\}\ knitr::opts\_chunk\$set(echo\ =\ TRUE)}

\hypertarget{teste}{%
\subsection{Teste}\label{teste}}

\hypertarget{code-for-reading-in-the-dataset-andor-processing-the-data}{%
\subsubsection{Code for reading in the dataset and/or processing the
data}\label{code-for-reading-in-the-dataset-andor-processing-the-data}}

\begin{verbatim}
temp <- tempfile()
download.file("https://d396qusza40orc.cloudfront.net/repdata%2Fdata%2Factivity.zip",temp)
data <- read.csv(unz(temp, "activity.csv"))
unlink(temp)
\end{verbatim}

\hypertarget{histogram-of-the-total-number-of-steps-taken-each-day}{%
\subsubsection{Histogram of the total number of steps taken each
day}\label{histogram-of-the-total-number-of-steps-taken-each-day}}

\begin{verbatim}
#Remove rows whose steps record is na
data1 <- data[which(!is.na(data$steps)),]

#Finds the unique records of the variable date
days <- unique(data1$date)

#Sum all records of same date
for (i in 1:length(days)){
  subset <- data1[which(data1$date == days[i]),]
  if(i == 1){
    df <- data.frame(days[1], sum(subset$steps))
  }else{
    df <- rbind(df, c(days[i], sum(subset$steps)))
  }
}

#Set column names
colnames(df) <- c("date","steps")

#Convert to integer
df$steps <- as.integer(df$steps)

#Histogram
library(ggplot2)

y <- ggplot(df, aes(x = steps)) 
a <- y + geom_histogram(bins = 30) + ylab("Frequency")
a
\end{verbatim}

\hypertarget{mean-and-median-number-of-steps-taken-each-day}{%
\subsubsection{Mean and median number of steps taken each
day}\label{mean-and-median-number-of-steps-taken-each-day}}

\begin{verbatim}
#Mean and median of all records of same date
for (i in 1:length(days)){
  subset <- data1[which(data1$date == days[i]),]
  if(i == 1){
    df <- data.frame(days[1], round(mean(subset$steps), 2), median(subset$steps))
  }else{
    df <- rbind(df, c(days[i],  round(mean(subset$steps), 2), median(subset$steps)))
  }
}

#Set column names
colnames(df) <- c("date","mean", "median")

#Convert to double
df$mean <- as.double(df$mean)

df
\end{verbatim}

\hypertarget{time-series-plot-of-the-average-number-of-steps-taken}{%
\subsubsection{Time series plot of the average number of steps
taken}\label{time-series-plot-of-the-average-number-of-steps-taken}}

\begin{verbatim}
#Finds the unique records of the variable interval
times <- unique(data1$interval)

#Mean of all records of same interval
for (i in 1:length(times)){
  subset <- data1[which(data1$interval == times[i]),]
  if(i == 1){
    df <- data.frame(times[1], round(mean(subset$steps), 2))
  }else{
    df <- rbind(df, c(times[i],  round(mean(subset$steps), 2)))
  }
}

#Set column names
colnames(df) <- c("interval","mean")

#Convert to double
df$mean <- as.double(df$mean)

y <- ggplot(df, aes(x = interval, y = mean))
b <- y + geom_line() # + geom_point()
b
\end{verbatim}

\hypertarget{the-5-minute-interval-that-on-average-contains-the-maximum-number-of-steps}{%
\subsubsection{The 5-minute interval that, on average, contains the
maximum number of
steps}\label{the-5-minute-interval-that-on-average-contains-the-maximum-number-of-steps}}

\begin{verbatim}
interval <- df[which(df$mean == max(df$mean)),]$interval
print(paste("The 5-minute interval that, on average, contains the maximum number of steps is", interval, "or", paste0(interval%/%100, ":", interval%%100)))
\end{verbatim}

\hypertarget{code-to-describe-and-show-a-strategy-for-imputing-missing-data}{%
\subsubsection{Code to describe and show a strategy for imputing missing
data}\label{code-to-describe-and-show-a-strategy-for-imputing-missing-data}}

\hypertarget{calculate-and-report-the-total-number-of-missing-values-in-the-dataset-i.e.-the-total-number-of-rows-with-nas}{%
\paragraph{\texorpdfstring{Calculate and report the total number of
missing values in the dataset (i.e.~the total number of rows with
\color{red}{\verb|NA|}NAs)}{Calculate and report the total number of missing values in the dataset (i.e.~the total number of rows with NAs)}}\label{calculate-and-report-the-total-number-of-missing-values-in-the-dataset-i.e.-the-total-number-of-rows-with-nas}}

\begin{verbatim}
print(paste("The total number of missing values in the dataset is", nrow(data[which(is.na(data$steps)),])))
\end{verbatim}

\hypertarget{devise-a-strategy-for-filling-in-all-of-the-missing-values-in-the-dataset.-the-strategy-does-not-need-to-be-sophisticated.-for-example-you-could-use-the-meanmedian-for-that-day-or-the-mean-for-that-5-minute-interval-etc.-create-a-new-dataset-that-is-equal-to-the-original-dataset-but-with-the-missing-data-filled-in.}{%
\paragraph{Devise a strategy for filling in all of the missing values in
the dataset. The strategy does not need to be sophisticated. For
example, you could use the mean/median for that day, or the mean for
that 5-minute interval, etc. Create a new dataset that is equal to the
original dataset but with the missing data filled
in.}\label{devise-a-strategy-for-filling-in-all-of-the-missing-values-in-the-dataset.-the-strategy-does-not-need-to-be-sophisticated.-for-example-you-could-use-the-meanmedian-for-that-day-or-the-mean-for-that-5-minute-interval-etc.-create-a-new-dataset-that-is-equal-to-the-original-dataset-but-with-the-missing-data-filled-in.}}

\begin{verbatim}
#Copy data to NewData
NewData <- data

#Fill in NA values by mean for that 5-minute interval
for (i in 1:length(times)){
  NewData[which(NewData$interval == times[i] & is.na(NewData$steps)),] = df[which(df$interval == times[i]),]
}
\end{verbatim}

\hypertarget{histogram-of-the-total-number-of-steps-taken-each-day-after-missing-values-are-imputed}{%
\subsubsection{Histogram of the total number of steps taken each day
after missing values are
imputed}\label{histogram-of-the-total-number-of-steps-taken-each-day-after-missing-values-are-imputed}}

\hypertarget{make-a-histogram-of-the-total-number-of-steps-taken-each-day-and-calculate-and-report-the-mean-and-median-total-number-of-steps-taken-per-day.}{%
\paragraph{Make a histogram of the total number of steps taken each day
and Calculate and report the mean and median total number of steps taken
per
day.}\label{make-a-histogram-of-the-total-number-of-steps-taken-each-day-and-calculate-and-report-the-mean-and-median-total-number-of-steps-taken-per-day.}}

\begin{verbatim}
#Finds the unique records of the variable date
days <- unique(NewData$date)

#Sum all records of same date
for (i in 1:length(days)){
  subset <- NewData[which(NewData$date == days[i]),]
  if(i == 1){
    NewDf <- data.frame(days[1], sum(subset$steps))
  }else{
    NewDf <- rbind(NewDf, c(days[i], sum(subset$steps)))
  }
}

#Set column names
colnames(NewDf) <- c("date","steps")

#Convert to integer
NewDf$steps <- as.integer(NewDf$steps)

#Histogram
library(ggplot2)

y <- ggplot(NewDf, aes(x = steps)) 
a <- y + geom_histogram(bins = 30) + ylab("Frequency")
a

print(paste("The new mean and median are", round(mean(NewData$steps),2),"and", paste0(median(NewData$steps), ", respectively.")))
\end{verbatim}

\hypertarget{do-these-values-differ-from-the-estimates-from-the-first-part-of-the-assignment}{%
\paragraph{Do these values differ from the estimates from the first part
of the
assignment?}\label{do-these-values-differ-from-the-estimates-from-the-first-part-of-the-assignment}}

\begin{verbatim}
if(mean(NewData$steps) != mean(data1$steps)){ answer <- "The mean has changed"}else{answer <- "The mean hasn't changed"}

if(median(NewData$steps) != median(data1$steps)){ answer2 <-"the median has changed"}else{answer2 <- "the median hasn't changed"}

print(paste(answer,"and", answer2))
\end{verbatim}

\hypertarget{what-is-the-impact-of-imputing-missing-data-on-the-estimates-of-the-total-daily-number-of-steps}{%
\paragraph{What is the impact of imputing missing data on the estimates
of the total daily number of
steps?}\label{what-is-the-impact-of-imputing-missing-data-on-the-estimates-of-the-total-daily-number-of-steps}}

\begin{verbatim}
if(sum(NewData$steps) - sum(data1$steps) > 0){ 
    answer <- paste("The number of steps increased", sum(NewData$steps) - sum(data1$steps), "steps.")
}else{
      answer <- paste("The number of steps decreased", sum(data1$steps) - sum(NewData$steps), "steps.")
}
print(answer)
\end{verbatim}

\hypertarget{panel-plot-comparing-the-average-number-of-steps-taken-per-5-minute-interval-across-weekdays-and-weekends}{%
\subsubsection{Panel plot comparing the average number of steps taken
per 5-minute interval across weekdays and
weekends}\label{panel-plot-comparing-the-average-number-of-steps-taken-per-5-minute-interval-across-weekdays-and-weekends}}

\hypertarget{create-a-new-factor-variable-in-the-dataset-with-two-levels-weekday-and-weekend-indicating-whether-a-given-date-is-a-weekday-or-weekend-day.}{%
\paragraph{Create a new factor variable in the dataset with two levels
-- ``weekday'' and ``weekend'' indicating whether a given date is a
weekday or weekend
day.}\label{create-a-new-factor-variable-in-the-dataset-with-two-levels-weekday-and-weekend-indicating-whether-a-given-date-is-a-weekday-or-weekend-day.}}

\begin{verbatim}
#install.packages("timeDate")
library(timeDate)

#Returns TRUE if the day is weekend, otherwise it returns FALSE
weekendDay <- isWeekend(as.Date(data1$date))
Data1 <-cbind(data1, weekendDay)

#Set variable in the dataset with two levels – “weekday” and “weekend” indicating whether a given date is a weekday or weekend day.
Data1[which(Data1$weekendDay == TRUE),]$weekendDay <- "weekend"
Data1[which(Data1$weekendDay == FALSE),]$weekendDay <- "weekday"

#Transform to factor variable
Data1$weekendDay <- as.factor(Data1$weekendDay)
\end{verbatim}

\hypertarget{make-a-panel-plot-containing-a-time-series-plot-of-the-5-minute-interval-x-axis-and-the-average-number-of-steps-taken-averaged-across-all-weekday-days-or-weekend-days-y-axis.}{%
\paragraph{Make a panel plot containing a time series plot of the
5-minute interval (x-axis) and the average number of steps taken,
averaged across all weekday days or weekend days
(y-axis).}\label{make-a-panel-plot-containing-a-time-series-plot-of-the-5-minute-interval-x-axis-and-the-average-number-of-steps-taken-averaged-across-all-weekday-days-or-weekend-days-y-axis.}}

\begin{verbatim}
#install.packages("ggpubr")
library(ggpubr)

typeOfDay <- c("weekday","weekend")

#Finds the unique records of the variable interval
times <- unique(Data1$interval)

#Mean of all records of same interval
for (i in 1:length(times)){
  for(j in 1:length(typeOfDay)){
    subset <- Data1[which(Data1$interval == times[i] & 
                            Data1$weekendDay == typeOfDay[j]),]
    if(j == 1){
      if(i == 1){
        df_weekday <- data.frame(times[1], round(mean(subset$steps), 2))
      }else{
        df_weekday <- rbind(df_weekday, c(times[i],  round(mean(subset$steps), 2)))
      }
    }else{
      if(i == 1){
        df_weekend <- data.frame(times[1], round(mean(subset$steps), 2))
      }else{
        df_weekend <- rbind(df_weekend, c(times[i],  round(mean(subset$steps), 2)))
      }      
    }
  }
}

#Set column names
colnames(df_weekday) <- c("interval","mean")
colnames(df_weekend) <- c("interval","mean")

#Convert to double
df_weekday$mean <- as.double(df_weekday$mean)
df_weekend$mean <- as.double(df_weekend$mean)

y <- ggplot(df_weekday, aes(x = interval, y = mean))
weekday <- y + geom_line() # + geom_point()

y <- ggplot(df_weekend, aes(x = interval, y = mean))
weekend <- y + geom_line() # + geom_point()


ggarrange(weekend, weekday, 
          labels = c("weekend", "weekday"),
          ncol = 1, nrow = 2)
\end{verbatim}

\end{document}
